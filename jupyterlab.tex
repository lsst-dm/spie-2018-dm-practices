\subsection{Case Study: Deploying releases to JupyterLab}
\label{sec:jupyterlab}

As our exploration of JupyterLab as the interactive notebook component
of the LSST Science Platform\cite{LSE-319} proceeds, we have realized the need for
automated deployments of the lab environment.
One of the build artifacts mentioned above is a Docker container that includes the stack.
Since the
JupyterLab environment is Kubernetes-ready, we have appended a phase to
the stack builds that starts with the Docker stack container and adds
the JupyterLab server and other Python modules we need, as well as the
user provisioning machinery and Lab startup scripts.  Thus, with each
nightly build of the stack, we also get a nightly build of the
JupyterLab environment wrapping that version of the stack.  In
JupyterHub, we have configuration directives that scan a Docker
repository looking for an image with tags in a particular format or set
of formats.  From this we can construct a menu of the most recent daily,
weekly, and release builds, and present a choice of stack versions to
the user at JupyterHub login time.

Automated deployment serves two
purposes: first, it makes it possible to immediately QA changes to the
lab environment itself, since the deployment can become part of a CI
process.  Second, it makes the process of standing up an environment for
a tutorial session or a conference extremely easy.  Our initial version
of the automated deployment tool relies on Google Compute Engine as the
back end, AWS Route 53 as the DNS provider, and either Github or CILogon
as the OAuth2 provider.  We will add more options as demand warrants.
The deployment tool provides a sequencer around \texttt{gcloud}, \texttt{kubectl}, and
\texttt{awscli}.  It is driven from a YAML file, from environment variables, or
from the command line.
The current version substitutes templated values from its configuration, and then runs \texttt{gcloud}
and \texttt{kubectl} to create the various components of the lab installation.
Future improvements may include a move to Helm charts rather than raw
Kubernetes YAML, a rewrite so that deployment becomes a Terraform
configuration rather than a python module, and/or integration of TLS
certificate management into the tool.
 A more detailed description of the LSST JupyterLab environment can be found in Ref.~\citenum{AS18-AS110-10707-16}.
