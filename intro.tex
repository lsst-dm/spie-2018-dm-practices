\section{INTRODUCTION}

The Data Management System (DMS)\cite{2015arXiv151207914J} for the Large Synoptic Survey Telescope (LSST) \cite{2008arXiv0805.2366I} has been under development since at least 2004\cite{2004AAS...20510811A}.
During that time a number of technologies have been adopted and our development practices have evolved as we transitioned from the research and development phase to construction.
The Data Management (DM) team is distributed, with representation from Princeton University, the University of Washington, the National Center for Supercomputing Applications at Urbana-Champaign, IPAC in Pasadena, the LSST Project Office in Tucson, and the SLAC National Accelerator Laboratory near Stanford University; along with some external contributions from CC-IN2P3 in Lyon, France.
As a uniform survey, LSST is highly dependent on its Data Management System, which, like the telescope and camera, has requirements that have historically been at or beyond the state of the art.
From the beginning LSST was recognized as needing a substantial software effort, and significant portions of both the design and development and the construction budgets have been devoted to that effort.
As a result, LSST has been able to hire full-time software engineering staff, including not only developers but also developer support staff.
Founding and growing the LSST Data Management team has thus been like creating a start-up software company.
Developers with experience in software engineering, not just scientific programming, were sought, several with experience in industry.
Best practices and tools used by open-source companies were adopted.
Evolution, not only of the code base but also of the development process itself, was embraced; developers were empowered to make changes.
Given the distributed team, it is important that comunication channels are open and easy to use and that our tools evolve as community standards evolve.
Being agile enough to be able to migrate from one tool to another during the lifetime of a project is key when the software, processes, and people, change over what will be a 25 year period once the 10-year survey completes.
For example, over the years we have migrated the codebase from Subversion to git (\S\ref{sec:subversion}); we have switched instant messaging from HipChat to Slack (\S\ref{sec:slack}); we have migrated continuous integration from Buildbot on our own hosts to Jenkins running in the cloud (\S\ref{sec:jenkins}); and we have moved documentation standards from Doxygen to Sphinx (\S\ref{sec:sphinx_tooling}).

In the following sections we describe the current development practices for LSST DM.
