\subsection{Development Model}\label{sec:development}

To enable concurrent development across all six LSST sites, the LSST development team uses a decentralized development model based on the Git version control system\footnote{\url{https://git-scm.com/}}.
[TODO: describe in terms of Agile values?]

\subsubsection{Version Control}\label{sec:git}\label{sec:subversion}

[TODO: why did LSST switch from Subversion to Git?]

\subsubsection{Code Organization}\label{sec:git_repositories}

The data management system is divided into several hundred individual Git repositories, hosted on GitHub\footnote{\url{http://github.com/lsst/}}.
Each repository corresponds to one code package, and all but a handful of repositories usually have at most one developer working on them at a time.
Isolating code in small repositories minimizes the need for developers to frequently download code updates, something the Git framework requires whenever developers make concurrent changes to the \texttt{master} branch of the same repository.
Representing each package by an independent repository also makes it easy to formalize and track inter-package dependencies using tools like EUPS (\autoref{sec:eups}).

\subsubsection{The LSST Workflow}\label{sec:dev_workflow}

Feature branches associated with Jira tickets.

Branch protection.

Rebase before merging.

\subsubsection{Comparison with other workflows}

git flow.

