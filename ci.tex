\section{Continuous Integration}

\subsection{Jenkins}
\label{sec:jenkins}

\subsubsection{Why Jenkins?}

The initial ``Continuous Integration'' system used for pre-merge testing of
science pipeline code was buildbotFOOTNOTE:URL-https://buildbot.net/.  While it
was able to accomplish the basic task of building branches from a git repo,
there were a number of drawbacks.
% \footnote{WRT the state of buildbot in 2014, the project appears to have made some improvements since that time}
A few of the issues were: the UI was spartan and difficult to navigate, no
integration with 3rd party authentication systems requiring manual management
of user accounts, bare-bones ``out of the box'' functionality without a useful
selection of publicly available plugins, and DM internal concern as to the long
term viability as there appeared be relatively few public/open source users
relative to competing CI/CD systems.

% should this be broken into a \item list?
After evaluating several potential CI/CD systems as a replacement,
JenkinsFOOTNOTE:URL-https://jenkins.io/ was selected for numerous reasons,
including: an open source core, it could be self hosted (the wall-clock build
time and memory requirements exceeded the limits of many of the commercial
hosted CI options, at that time), it offered an improved UI over buildbot,
there was a pre-existing plugin to use github
oauthFOOTNOTE:URL-https://plugins.jenkins.io/github-oauth, a healthy extension
ecosystem with many useful plugins, apparent popularity with self-hosted
open source projects, and an active core project.

\subsubsection{Configuration + Deployment}

The Jenkins core and various plugins need to be version managed and configured.
Although there is currently a major effort underway to add native
``configuration by
code''FOOTNOTE:URL-https://github.com/jenkinsci/configuration-as-code-plugin to
the jenkins core, this is not yet considered production ready and did not exist
at the time DM was transitioning away from buildbot.

PuppetFOOTNOTE:URL-https://puppet.com/ was selected as a configuration
management tool as, at that time, it had the most sophisticated jenkins
management abilities among the CM tools surveyed via the
\texttt{puppet-jenkins}FOOTNOTE:URL-https://github.com/voxpupuli/puppet-jenkins
module.  Non-trivial improvements have been contributed by DM staff to this
module
FOOTNOTE:URL-https://puppet.com/presentations/puppet-vs-jenkins-tale-types-and-providers
in order to make it more suitable for managing a jenkins deployment.

Configuration of Jenkins jobs is handled via the
\texttt{job-dsl}FOOTNOTE:URL-https://plugins.jenkins.io/job-dsl plugin.  This
enables a \texttt{groovy}FOOTNOTE:url-http://www.groovy-lang.org/ based DSL and a special job type that
will ``seed'' jenkins jobs from a git repository that contains \texttt{job-dsl}
script(s).  The seed job itself is maintained as \texttt{xml} that is manually
installed by puppet.  The result is that all jobs configuration is managed via
a SCM and no manual configuration via the jenkins UI is required to
add/delete/change or stand up a testing environment.

% should any of the jenkins related repos be mentioned?
% the production jenkins deployment is not publically accessible

% add discussion of jenkins pipeline

\subsubsection{Evolving Usage}

Over time, usage of jenkins has evolved from being used solely for pre-merge
``CI'' testing to automation of number of common tasks.  Among various sundry
tasks, it is being used to build docker images, update local software mirrors,
schedule regular backup.  Perhaps most notably, it is being used to a drive a
`Continuous Deployment'' workflow for completely automated nightly and weekly
release/publication of science pipelines codes(\ref{sec:scipipe-deploy}).

\subsection{Travis-CI}
\label{sec:travis-ci}

Travis for simpler items (yaml file validation, flake8 tests)
