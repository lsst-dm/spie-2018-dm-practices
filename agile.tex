\section{Agile Development}\label{sec:jira_ticket}

LSST data management follows a form of cyclic development often referred to as {\emph Agile} methodology. It is also beholden to organizations such as NSF which require a more traditional approach to project development such as the Earned Value Management System (EVMS).
We have presented this from both  the ESA and LSST perspective in SPIE previously  \cite{2014SPIE.9150E..1EG}.
In 2016  we presented a more complete approach for LSST to the problem of Agile in the earned value world \cite{2016SPIE.9911E..0NK}, here we provide just a brief update on that paper.

\subsection{Management}
\cite{DMTN-020} provides a guide to the mechanisms underpinning LSST Data Management’s approach to project management, the reader is referred to to this for gory details as required.
Each institution in the DM team is still typically
responsible for 1 or more Level 2 WBS elements, and each institution has a Technical/Control Account Manager
(T/CAM) responsible for planning, estimating, monitoring and EVM reporting for that Level 2 WBS element.

\subsection{Basic Assumptions}
The Project assumes that a full-time individual works for a total of
1,800 hours per year: this figure is \emph{after} all vacations, sick
leave, etc are taken into account. Staff appointed to ``developer''
positions are expected to devote this effort directly to LSST.

Appointment as a ``scientist'' includes a 20\% personal research time
allowance. That is, scientists are expected to devote 1,440 hours per
year to LSST, and the remainder of their time to personal research.

Our base assumption is that 30\% of an individual's LSST time (i.e. 540 hours/year for a developer, 432 hours/year for a scientist) are devoted to overhead for regular meetings\footnote{``Meetings'' include, for example, scheduled weekly team meetings, stand-ups, etc; major conferences or project meetings involving preparation, travel time, etc should be scheduled in advance and allocated Story Points}, ad-hoc discussions and other interruptions.
This is similar to the standard {\emph Agile} discount, however in the earned value world that must be accounted for and it is considered Level of Effort (LOE)



Epics, stories, P6, milestones, EVMS.
How do we use product owners?
Test specifications and requirements.

How has this evolved since Kantor et al SPIE 2016? (kind of important that we link this SPIE paper to previous SPIE paper on Agile process).

