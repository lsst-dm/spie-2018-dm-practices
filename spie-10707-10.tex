\documentclass[]{spie}  %>>> use for US letter paper
%\documentclass[a4paper]{spie}  %>>> use this instead for A4 paper
%\documentclass[nocompress]{spie}  %>>> to avoid compression of citations

\renewcommand{\baselinestretch}{1.0} % Change to 1.65 for double spacing

\def\procspie{Proc.\ SPIE} % Proceedings of the SPIE

\usepackage{amsmath,amsfonts,amssymb}
\usepackage{graphicx}
\usepackage[colorlinks=true, allcolors=blue]{hyperref}

\title{LSST Data Management Software Development Practices and Tools}

\author[a]{Tim~Jenness}
\author[a]{Frossie~Economou}
\author[b]{Krzysztof~Findeisen}
\author[c]{Fabio~Hernandez}
\author[a]{Josh~Hoblitt}
\author[a]{K.~Simon~Krughoff}
\author[d]{Kian-Tat~Lim}
\author[e]{Robert~H.~Lupton}
\author[d]{Fritz~Mueller}
\author[a]{William~O'Mullane}
\author[b]{Russell~Owen}
\author[f]{Stephen~R.~Pietrowicz}
\author[e]{Pim~Schellart}
\author[a]{Jonathan~Sick}
\author[b]{John~Swinbank}
\author[a]{Adam~Thornton}
\author[a]{Jeffrey~Kantor}
\author[b]{John~K.~Parejko}
\author[e]{Paul~Price}


\affil[a]{LSST Project Office, 950 N.\ Cherry Avenue, Tucson, AZ 85719, USA}
\affil[b]{University of Washington, Dept.\ of Astronomy, Box 351580, Seattle, WA 98195, USA}
\affil[c]{CNRS, CC-IN2P3, 21 avenue Pierre de Coubertin, CS70202, 69627 Villeurbanne cedex, France}
\affil[d]{SLAC National Accelerator Laboratory, 2575 Sand Hill Rd, Menlo Park, CA 94025, USA}
\affil[e]{Department of Astrophysical Sciences, Princeton University, Princeton, NJ 08544}
\affil[f]{NCSA, University of Illinois at Urbana-Champaign, 1205 W.\ Clark St.\ Urbana, IL 61801}



\authorinfo{Further author information: (Send correspondence to T.J.)\\T.J.: E-mail: tjenness@lsst.org,\\  W.O'M: E-mail: womullan@lsst.org}

% Option to view page numbers
\pagestyle{empty} % change to \pagestyle{plain} for page numbers
\setcounter{page}{1} % Set start page numbering at e.g. 301

\begin{document}
\maketitle

\begin{abstract}
The Large Synoptic Survey Telescope (LSST) is an 8.4m optical survey telescope being constructed on Cerro Pach\'on in Chile.
The data management system being developed must be able to process the nightly alert data, 20,000 expected transient alerts per minute, in near real time, and construct annual data releases at the petabyte scale.
The development team consists of more than 90 people working in six different sites across the US developing an integrated set of software to meet the LSST requirements.
In this paper we discuss our agile software development methodology and our API and developer decision making process.
We also discuss the software tools that we use for continuous integration and deployment.
\end{abstract}

% Include a list of keywords after the abstract
\keywords{Agile Development, LSST, Distributed Development, Continuous Integration}


\section{INTRODUCTION}

The data management system\cite{2015arXiv151207914J} for the Large Synoptic Survey Telescope\cite{2008arXiv0805.2366I} has been under development since at least 2004\cite{2004AAS...20510811A}.
During that time a number of technologies have been adopted and our development practices have evolved as we transitioned from the research and development phase to construction.
The Data Management (DM) team is distributed, with representation from Princeton University, the University of Washington, the National Center for Supercomputing Applications at Urbana-Champaign, IPAC in Pasadena, the LSST Project Office in Tucson, and the SLAC National Accelerator Laboratory near Stanford University; along with some external contributions from CC-IN2P3 in Lyon, France.
Given this, it is important that comunication channels are open and easy to use and that our tools evolve as community standards evolve.
Being agile enough to be able to migrate from one tool to another during the lifetime of a project is key when the software, processes, and people, change over what will be a 25 year period once the 10-year survey completes.
For example, over the years we have migrated the codebase from Subversion to git; we have switched instant messaging from HipChat to Slack; we have migrated continuous integration from builbot on our own hosts to Jenkins running in the cloud; and we have moved documentation standards from Doxygen to Sphinx.

In the following sections we describe the current development practices for LSST DM.

\section{Agile Development}\label{sec:jira_ticket}

LSST data management follows a form of cyclic development often referred to as {\emph Agile} methodology. It is also beholden to organizations such as NSF which require a more traditional approach to project development such as the Earned Value Management System (EVMS).
We have presented this from both  the ESA and LSST perspective in SPIE previously  \cite{2014SPIE.9150E..1EG}.
In 2016  we presented a more complete approach for LSST to the problem of Agile in the earned value world \cite{2016SPIE.9911E..0NK}, here we provide just a brief update on that paper concentrating more on the Agile aspects.


\subsection{Management}
\cite{DMTN-020} provides a guide to the mechanisms underpinning LSST Data Management’s approach to project management, the reader is referred to to this for gory details as required.
Each institution in the DM team is still typically
responsible for 1 or more Level 2 WBS elements, and each institution has a Technical/Control Account Manager
(T/CAM) responsible for planning, estimating, monitoring and EVM reporting for that Level 2 WBS element.
The detailed management plan is in \cite{LDM-294}.

\subsection{Basic Assumptions}
The Project assumes that a full-time individual works for a total of
1,800 hours per year: this figure is \emph{after} all vacations, sick
leave, etc are taken into account. Staff appointed to ``developer''
positions are expected to devote this effort directly to LSST.

Appointment as a ``scientist'' includes a 20\% personal research time
allowance. That is, scientists are expected to devote 1,440 hours per
year to LSST, and the remainder of their time to personal research.

Our base assumption is that 30\% of an individual's LSST time (i.e. 540 hours/year for a developer, 432 hours/year for a scientist) are devoted to overhead for regular meetings\footnote{``Meetings'' include, for example, scheduled weekly team meetings, stand-ups, etc; major conferences or project meetings involving preparation, travel time, etc should be scheduled in advance and allocated Story Points}, ad-hoc discussions and other interruptions.
This is similar to the standard {\emph Agile} discount, however in the earned value world that must be accounted for and it is considered Level of Effort (LOE)


\section{Long Term Planning}
\label{sec:long-term-plan}

The plan for the duration of construction is embodied in:

\begin{enumerate}
\item
  A series of \emph{planning packages}, which describe major pieces of
  technical work. Planning packages are associated with concrete, albeit
  high-level, deliverables (in the shape of milestones), and have
  specific resource loads (staff assignments), start dates, and
  durations. The entire DM system is covered by around 100 of these
  planning packages.
\item
  \emph{Milestones} represent the delivery or availability of specific
  functionality. Each planning package culminates in a milestone, and
  may contain other milestones describing intermediate results.
\end{enumerate}

Planning packages are defined at the fourth level of the Work Breakdown Structure (WBS).

All WBS elements are related to the set of Data Management Products embodied in \citep{LDM-148} a high level summary of which is given in
Figure \ref{fig:prods}. Each product has a product owner to guide the agile development, this is frequently one of the Data Management Scientists.

\begin{figure}[htbp]
        \begin{center}
                 \includegraphics[height=19cm]{ProductTree}
                 \caption{DM product tree. \label{fig:prods}- there are over 200 products, this tree is to convey and idea of the products and is truncated to make it somewhat legible.
         }

         \end{center}
 \end{figure}

During the cycle planning process  effort is drawn from the budget embodied in the planning packages to generate the cycle plan, described in terms of epics in Jira.
Each epic itself has a particular budget.
This budget is subtracted from that available in the planning package at the point when the epic is defined.
Team members then add specific stories to these epics.

The Jira system is synchronized with the Primavera project management tool every three months using in house tools.

In order for the DM system to reach its science goals, new algorithmic or engineering approaches must sometimes be researched.
It is appropriate to budget time for this research work in planning packages.
Areas where successful delivery of the DM system is dependent on speculative research are a source of risk: where possible, the plan  also provides for a fallback option to be taken when research objectives are not achieved.
This may also lead to an entry in the risk register.



\section{Reporting }
Though the EVMS system is reasonable good for reporting it does not report well on Agile progress. Since  \citep{2016SPIE.9911E..0NK} we have undertaken a set of test driven milestones to show progress in Data Management.

Epics, stories, P6, milestones, EVMS.
How do we use product owners?
Test specifications and requirements.

How has this evolved since Kantor et al SPIE 2016? (kind of important that we link this SPIE paper to previous SPIE paper on Agile process).


\section{Model Based Systems Engineering}

\begin{itemize}
  \item MagicDraw (used to be EA)
  \item Requirements LSE-61 to subsystem requirements models.
  \item Components
  \item Do we use other diagrams regularly in DM?
\end{itemize}

\section{Software Development}

\subsection{Development Model}\label{sec:development}

To enable concurrent development across all six LSST sites, the LSST development team uses a decentralized development model based on the Git version control system\footnote{\url{https://git-scm.com/}}.
[TODO: describe in terms of Agile values?]

\subsubsection{Version Control}\label{sec:git}\label{sec:subversion}

[TODO: why did LSST switch from Subversion to Git?]

\subsubsection{Code Organization}\label{sec:git_repositories}

Lots of small repositories.

\subsubsection{The LSST Workflow}\label{sec:dev_workflow}

Feature branches associated with Jira tickets.

Branch protection.

Rebase before merging.

\subsubsection{Comparison with other workflows}

git flow.


\subsection{Code Quality}

LSST DM is responsible for writing and supporting a large code base.
As of April 2018, the Qserv database software\cite{2011Wang:2011:QDS:2063348.2063364} is about 100,000 lines of C++, Firefly has about the same amount of JavaScript\cite{2016SPIE.9913E..0YR}, and the Science Pipelines software\cite{2018PASJ...70S...5B} is approximately 290,000 lines of Python and 225,000 lines of C++.\footnote{Line counts include comments but not blank lines. Python interfaces are implemented using \texttt{pybind11} and that is counted as C++ code. Science pipelines software is defined as the \texttt{lsst\_distrib} metapackage and does not include code from third party packages.}

% starlink_ast is exempted. Calculation was done on cleaned directories
% on 2018-04-02.
% perl ~/Downloads/cloc-1.72.pl $(cat manifest.txt  | awk '{print $1}')
%
% --------------------------------------------------------------------------------
% Language                      files          blank        comment           code
% --------------------------------------------------------------------------------
% Python                         1771          54192          95609         192982
% C++                             859          21507          29472         108609
% C/C++ Header                    555          14937          46781          42296

Code quality is hard to define and notoriously difficult to tackle when hundreds of thousands of lines of code are to be supported.
In DM we have a coding standard\cite{devguide} enforced with tooling where possible.
We ensure  that warnings from static analyzers are dealt with and where possible have the build fail e.g., if a \texttt{flake8} warning is issued;
enforce a specific C++ standard (C++14) by configuring the C++ compiler to fail if it is violated;
run unit tests for every commit, which also generates code coverage and test statistics;
issue verbose warnings, including Python's \texttt{DeprecationWarning}, to warn us of any possible future issues;
and run regular integration tests while tracking metrics of their data products and performance to ensure that we are improving with time.

\subsubsection{Coding Standards}
\label{sec:coding-standards}

Over the years, the DM team have defined extensive coding standards\cite{devguide} covering C++ and Python.
This includes variable naming conventions, indenting policies and also how to document code.
We use \texttt{numpydoc}\cite{numpydoc} for Python, Doxygen\cite{doxygen} for C++, and JSDoc\cite{jsdoc} for JavaScript, and we automate API documentation generation.

In 2017 we significantly simplified our Python standard by defining it relative to the PEP\,8 standard\cite{pep8}.
We could not fully adopt PEP\,8 since we had to balance the benefits of automated code linting versus the cost of changing tens of thousands of lines of code.
In particular, the variable and method naming standard had been to use camel case everywhere, and it was felt to be too disruptive to change all the code, especially since to minimize confusion we would have to also modify the C++ code that is called from Python.
There are two other areas where we are not compliant with PEP\,8.
Firstly, we disagree with the whitespace rules concerning binary operators and therefore disable this test.
Secondly, we feel that an 80 character line limit is too short and we continue to use a 110 character limit for code.
We did experiment with switching to 80 characters for one package but did not like the result.
For documentation and comments we  adopted the 75  character limit of Numpydoc.\cite{numpydoc}
PEP\,8 does specify their line lengths should be different than for code, but at this time there is no tooling in place to warn of non-compliance.
We have written and submitted a patch to the \texttt{pycodestyle} tool to implement a check for this, which has yet to be  accepted.
The \texttt{flake8} tool is used to validate code compliance, and we are happy with the results.
Once implemented, code reviews no longer waste time worrying about whether a space should be in a particular place and can now focus on the code functionality.

Based on the positive  \texttt{flake8} experience on the Python code base, we investigated options for C++.
Our code layout and formatting rules were updated to be based on the Google C++ Style Guide\cite{googlestyle} with a few minor modifications, such as 4 space indentation and a 110 character line limit to maintain consistency with our Python coding standards.
This allowed us to easily enforce these rules using the \texttt{clang-format}\cite{clangformat} tool with minimal additional configuration, though developers are also permitted to format their code manually or with any tool consistent with our standards.
We are also investigating C++ linting with \texttt{clang-tidy}\cite{clangtidy}. Although higher runtime and configuration cost at this time make it unlikely to be used for automatic linting at pull-request time, it is still a valuable tool in our workflow.
Because it is based on the Clang C++ compiler \texttt{clang-tidy} has access to all the requisite information needed to find common causes for bugs using static analysis, identify things that are hard to find by eye (such as whether \texttt{0} represents a pointer value to be replaced by \texttt{nullptr} and whether a member function overrides a virtual function from a base class and should be marked with \texttt{override}), and automatically update a whole codebase to use features from modern C++.
% One valuable check for \texttt{clang-tidy} to perform at pull-request-time might be enforcing identifier (for example, classes, functions and variables) naming rules.

\subsubsection{Language Standards}

We do not require a specific C++ compiler to build the DM software, but we do require compliance
with the  ISO C++14 standard\cite{cpp14}. We configure the Clang and GCC compilers to fail if the code is non-compliant.
There is still code that uses C++11 and older style but we are slowly updating the code as we encounter constructs that can be improved or simplified.
We are monitoring the support of the C++17 standard in common compilers.

%The DM Python code was begun in the era of Python 2.3, and was targeting Python 2.7 by 2016.
Given the commissioning timeline for LSST and the support schedule for Python 2.7,
it is clear that Python 3 should be our ultimate target. We completed the migration to support both Python 3 and 2.7 during 2017\cite{2016arXiv161100751J}.
In April 2018 we dropped support for Python 2 and have now standardized on Python 3.6\cite{2017arXiv171200461J}.
This will allow us to use all Python 3 features for the first time.

\subsubsection{Unit Testing}

Unit testing is critically important when writing code.
In DM we do almost all our testing from Python using the \texttt{unittest} module and we test the C++ code by using the Python bindings.
Tests are run using \texttt{pytest}\cite{pytest}, including several useful features provided via plugins.
We run the tests in a multi-process mode, based on the number of cores on the system, and all code is tested with \texttt{flake8} for compliance.
The output from the tests, including failure information, test count, and execution time, are written in JUnit XML format.  This output is parsed by the Jenkins system to provide a test report to the relevant developer.
We are also investigating the code coverage plugin and intend to record metrics of code coverage and test execution time as a function of time for long term trending analysis.
Tests are run with full Python warnings enabled, including \texttt{DeprecationWarning}, and we expect these warnings to be fixed.
Newer versions of \texttt{numpy} began warning when \texttt{NaN}s are used in numerical calculations. We have had to disable those warnings on a case-by-case basis once we confirm that \texttt{NaN} is a reasonable value for that particular code path; we decided not to globally disable all the numeric warnings.

\subsubsection{Algorithmic Performance Tracking}

At its most basic, the LSST DM team has to deliver algorithms that can calculate results for photometry and astrometry with sufficient accuracy to meet the overall science requirements of the LSST\cite{LPM-17}.
Additionally, we have fixed timing and computational performance requirements, and our algorithms must fit within the expected compute resources.
It is important for us to know quickly if an algorithm suddenly gets worse during code development.
We have developed a set of Key Performance Metrics that we calculate on a weekly basis using precursor data from instruments on CFHT and Subaru.
These metrics are logged with the SQuaSH system\cite{SQR-009} to enable scientists to examine trends and look for anomalies and to allow developers to test their development branches to ensure that there aren't unexpected regressions in algorithmic performance.
We have found that tracking metrics is not quite enough because of the complex interplay between datasets and algorithms.
Care must be taken to evaluate the metrics with knowledge of the context, rather than relying on simple thresholds.
When unexpected changes in metrics are discovered, domain experts and DM scientists work together to understand why the change occurred.

\subsection{EUPS}

Philosophy of eups

Implementation overview.

How we use EUPS.

Eupspkg.

\subsection{Building the software}

With more than one hundred distinct git repositories, building a coherent version of the LSST Science Pipelines software is not as simple as it would be with a large single repository containing all the code.
We use EUPS to specify the dependencies between packages and the \texttt{ups} directory contains those dependencies and a description of how the code should be accessed from other packages.

\subsubsection{Software Packaging}

There are four types of software that are required to be handled in order to install the LSST DM Science Pipelines software packages.
The first type are the system prerequisites that we need to have available on the system: this can include compilers, and build support tools such as \texttt{cmake}.
The second type is Python and the set of standard Python packages.
Although the software should be able to be built with any valid Python, for LSST development and testing we always install our own Anaconda and install a set of Python prerequisites including \texttt{numpy}, \texttt{matplotlib}, and \texttt{astropy}\cite{2018arXiv180102634T}.

The third type of packages are EUPS ``tar and patch'' packages of third-party code.
These packages consist of a (compressed) tar file of the upstream distribution, an optional directory of patches to be applied after the source code has been extracted, and a \texttt{ups} directory containing a table file for dependency management along with a file named \texttt{eupspkg.cfg.sh} that provides instructions for overriding the default build and install behavior.
The \texttt{eupspkg} command has \texttt{prep}, \texttt{config}, \texttt{build}, and \texttt{install} sub-commands for preparing the package, configuring it, building it, and installing it respectively, and by default knows how to build Python packages, and packages using autoconf, SCons, and make.
This approach to thirdparty packages allows us to know what patches we have applied and has the benefit of giving us direct control over a specific version that we wish to build against.
We also have the concept of ``stub'' packages, whose job it is to validate that a prerequisite Python package is installed and has a supported version.
This also allows our own packages to explicitly require these packages be available for dependency tracking.

In a few cases, where the third-party software is available on GitHub and we are closely affiliated with upstream, we fork the code repository and use a branch named \texttt{lsst-dev} for local modifications such as adding EUPS support.
For Science Pipelines we use this technique for the AST library\cite{2016A&C....15...33B} for handling coordinate transformations.

The LSST EUPS-based software packages themselves use a standard layout with SCons\cite{2005Scons1377085} being used to build each package.
We have written extensions to SCons to support our use of C++ with Pybind11 wrappers, and Python code tested with pytest.

Support packages from the Science Quality and Reliability Engineering group, use industry standard packaging systems based on the language in the package, with Python packages being distributed through PyPI and dependencies managed in the standard Python manner.

\subsubsection{Building from git}

With all the prerequisites installed we needed a way to retrieve all the source code from GitHub and to work out which packages should be built and in what order.
We have implemented this by writing two support packages that use EUPS for dependency management and \texttt{eupspkg} for builds.
The \texttt{lsstsw} package was originally written purely to support continuous integration services but is now the default choice for developers who wish to build the most up to date version of the software from source.
There is a shell script to install the local build system by installing Python and EUPS and there is a Python program for doing a full build.
This \texttt{rebuild} program is a wrapper around the \texttt{lsst\_build} package.
There are two key phases to doing a build.
In the first phase all the source is cloned from GitHub and the correct git reference is checked out based on the user request (usually you list the feature branch name that you are testing).
We have a master list in a GitHub repository that maps product name to git repository and the command can be made to refresh this list every time it runs.
Once the code has been cloned and the source repositories configured properly, the second phase is to build the code in dependency order and install it using EUPS tagged with the current build number.
The build tag can then be used to setup the products from that specific build.

\subsubsection{Building from a source distribution}

Not everyone needs to be able to build the cutting edge version of the code and for those users we publish packages to an EUPS server and EUPS can download the packages associated with a public EUPS tag.
This is usually a weekly, for example \texttt{w\_2018\_13}, or an official release, such as \texttt{v15\_0}.
These tags are distinct from git tags, the corresponding version of which would be, say, \texttt{w.2018.13} in each repository (see Fig~\ref{fig:commitlog}).

\subsubsection{Future Considerations}

We have considered creating a git repository similar to the current \texttt{lsst\_distrib} metapackage, containing git submodules to the separate git repositories, and this is how we implement a single git repository of all the DM documentation.
This would have the advantage of being able to check out specific versions of all the desired packages, and whilst the updates to the metapackage could be automated, this comes at the added complication of submodules.
We are also investigating the adoption of true semantic versioning\cite{semver} for all our modules and conversion of the packaging to standard Python packages that can be \texttt{pip}-installable (see for example the discussion in Ref.~\citenum{2016SPIE.9913E..0GJ}) but this is not yet past a proof of concept stage.


\section{Communication}

\subsection{Slack}

Description of the types of communications that happen on Slack

Collaboration involvement

\subsubsection{Chat Ops}

DM-SQuaRE is using a Slack chatbot called ``sqrbot'' to make some tasks
easier. Currently it performs a range of functions, from returning the
status of various infrastructure machines to creating technotes to
monitoring whether metrics in our processing stack have changed between
CI builds. What these tasks have in common is that they were frequently
requested actions that required someone being asked for help and having
to interrupt their own work to perform a task. With sqrbot, the
requestor can simply ask sqrbot in a Slack channel for the information
(or creation of a technote skeleton, etc.) and get immediate gratification.

The basic architecture is simple: sqrbot is simply a collection of hubot
scripts running as a Slack bot, which in turn drive microservices,
written in Python and implemented using the Flask framework.  These
microservices have an API that responds with JSON, so the job of sqrbot
is simply to accept commands, create appropriate HTTP transactions, and
then reformat the output into a conversational format.  The whole
assembly runs in a Kubernetes cluster.


\subsection{Community forum}

Description of purpose and usage.

\subsection{Confluence}

Transient information. Not used to document processes or for reports.

* Planning hack sessions?

* Integrating Jira tickets into confluence pages to indicate progress.

Meeting minutes

* Short term actions -- migrate to Jira if long term.


\section{Documentation}

\subsection{Types of Documentation}

\subsubsection{Project documentation}

* Controlled

* Technical notes

* texmf: class file for latex. shared bibtex, even for rst.

\subsubsection{User documentation}

How user documentation is different than project documentation

Examples

* Developer.lsst.io (internal ‘user documentation’)

* Pipelines.lsst.io (example of user documentation for a product)

\subsection{How we build documentation}
\label{sec:doc_builds}

Besides creating documentation content, we invest in infrastructure that helps us more effectively create and serve that documentation. 
Overall, we use a ``docs-like-code'' approach.\cite{Gentle:2017}
This means, broadly, that we write documentation like we write code by using plain text source formats, Git and GitHub for version control and collaboration, and continuous integration servers for both validating the content and deploying it to web servers.
For a code development organization like ours, this has several advantages.
By using the same development processes for both code and documentation, our developers need minimal additional training and documentation. 
In the particular case of software user documentation, the docs-like-code approach ensures that documentation is versioned with the software since both reside in the same Git repository.
The docs-like-code approach is also convenient for building custom tooling to automate documentation processes and integrate with the code base.

\subsubsection{Versioned documentation delivery}
\label{sec:ltd}

A critical component of our docs-like-code system is the documentation hosting service, which we call \textit{LSST the Docs}\cite{SQR-006} in reference to the popular service Read the Docs,\footnote{\url{https://readthedocs.org}}.
We designed \textit{LSST the Docs} around three goals: reliable scalability, support for versioned documentation, and flexible integration with continuous delivery workflows.

LSST documentation is statically rendered at build time, stored in the cloud (Amazon S3) and delivered via a content distribution network (Fastly).
This architecture is arbitrarily scalable, and avoids the need to maintain servers and applications to dynamically render HTML.
The static documentation approach also lends itself to \emph{versioned} documentation.
\textit{LSST the Docs} uses URL path prefixes to deliver documentation for different versions of the same project.
In fact, we often deploy draft versions of documentation based on Git branches to support code reviews.

Finally, \textit{LSST the Docs}'s modular architecture make it easy to integrate into many kinds of software continuous integration pipelines. 
Generally all documentation projects hosted on \textit{LSST the Docs} have some form of continuous integration server.
Our LaTeX projects and technical notes are built with Travis CI, triggered automatically by GitHub activity.
Documentation for the larger, and more complex, LSST Stack software is built on a Jenkins CI server.
In each case, the continuous integration pipeline uses a client to upload built documentation to \textit{LSST the Docs}.
\textit{LSST the Docs} does not provide is own documentation build service.
This is in distinct contrast to other static documentation hosting services, like Read the Docs, and allows the necessary flexibility to let software and documentation projects define their own build environments.

In the next sections we describe the build systems for specific types of DM documentation that are deployed with \textit{LSST the Docs}.

\subsubsection{LaTeX tooling for project documentation}
\label{sec:latex_tooling}

We write most project documentation (add reference) with LaTeX.
To standardize these documents, and ease maintenance, we use a lsst-texmf\footnote{\url{https://github.com/lsst-texmf}} package (based on LaTeX classes developed for Gaia) that provide a document class, styles, and bibliography styles.
The package also contains a common list of acronyms to populate document glossaries.
Additionally, we package common BibTeX bibliography files in the lsst-texmf repository.
Every DM LaTeX document is expected to upstream all of its citations into these common bibliography files.
We also maintain a common repository of images,\footnote{\url{https://github.com/lsst-dm/images}} though in a separate Git repository. 
Our LaTeX documents typically incorporate these images as a Git submodule.

We practice continuous integration and delivery of our LaTeX documents.
We write LaTeX documents on GitHub (each distinct document is a distinct Git repository).
When changes are pushed to a branch on GitHub, Travis CI builds the document.
Within the Travis CI environment, we compile the LaTeX document inside a Docker container that includes Texlive and lsst-texmf.
We've found that this containerized approach generally improves build times.
After building the document, the Travis CI job uploads the versioned document (corresponding to a Git branch or tag) to \textit{LSST the Docs}.
Since LaTeX documents compile to PDF, we create HTML landing pages that wrap the PDF and provide metadata to readers.\footnote{\url{https://github.com/lsst-sqre/lander}}

Developing a document on GitHub is convenient for reviewing and approving change-controlled project documentation.
When changes on the \texttt{master} branch are ready to become the baseline, we create a release branch to incorporate feedback from the change control board.
Once the change control board approves a document, we tag that version.
LSST archives baselined documents, as well as intermediate drafts, in our Xerox Docushare repository.

\subsubsection{Sphinx-based user documentation}
\label{sec:sphinx_tooling}

For user guides, such as DM's Developer Guide and the LSST Science Pipelines documentation, we use Sphinx to generate static websites that we deploy to \textit{LSST the Docs}.
Sphinx is ideal because it integrates well with our Python codebase.
For example, docstrings within our Python code are written in the Numpydoc\cite{numpydoc} format, which is incorporated in our codebase with the Numpydoc toolchain that is wrapped with extensions from the Astropy\cite{2018arXiv180102634T} project.
Beyond Numpydoc we take further advantage of extensions developed by Sphinx's open source community.
For example, the breathe extension pulls data about our C++ APIs from Doxygen, allowing us to generate C++ API references with Sphinx.
We are also developing our own custom extensions, specific to LSST documentation.
These are being developed in our Documenteer Python package.

For software projects, we incorporate the Sphinx site directly into the software's codebase.
Since our EUPS-managed software stacks are composed of many Git repositories, we maintain documentation in each repository, but combine that documentation at build-time into a single documentation website.


\section{Decision Making Process}

\subsection{RFCs}

\subsection{Change Control Board}

\section{Continuous Integration}

\subsection{Jenkins}
\label{sec:jenkins}

\subsubsection{Why Jenkins?}

The initial ``Continuous Integration'' system used for pre-merge testing of
science pipeline code was buildbotFOOTNOTE:URL-https://buildbot.net/.  While it
was able to accomplish the basic task of building branches from a git repo,
there were a number of drawbacks.
% \footnote{WRT the state of buildbot in 2014, the project appears to have made some improvements since that time}
A few of the issues were: the UI was spartan and difficult to navigate, no
integration with 3rd party authentication systems requiring manual management
of user accounts, bare-bones ``out of the box'' functionality without a useful
selection of publicly available plugins, and DM internal concern as to the long
term viability as there appeared be relatively few public/open source users
relative to competing CI/CD systems.

% should this be broken into a \item list?
After evaluating several potential CI/CD systems as a replacement,
JenkinsFOOTNOTE:URL-https://jenkins.io/ was selected for numerous reasons,
including: an open source core, it could be self hosted (the wall-clock build
time and memory requirements exceeded the limits of many of the commercial
hosted CI options, at that time), it offered an improved UI over buildbot,
there was a pre-existing plugin to use github
oauthFOOTNOTE:URL-https://plugins.jenkins.io/github-oauth, a healthy extension
ecosystem with many useful plugins, apparent popularity with self-hosted
open source projects, and an active core project.

\subsubsection{Configuration + Deployment}

The Jenkins core and various plugins need to be version managed and configured.
Although there is currently a major effort underway to add native
``configuration by
code''FOOTNOTE:URL-https://github.com/jenkinsci/configuration-as-code-plugin to
the jenkins core, this is not yet considered production ready and did not exist
at the time DM was transitioning away from buildbot.

PuppetFOOTNOTE:URL-https://puppet.com/ was selected as a configuration
management tool as, at that time, it had the most sophisticated jenkins
management abilities among the CM tools surveyed via the
\texttt{puppet-jenkins}FOOTNOTE:URL-https://github.com/voxpupuli/puppet-jenkins
module.  Non-trivial improvements have been contributed by DM staff to this
module
FOOTNOTE:URL-https://puppet.com/presentations/puppet-vs-jenkins-tale-types-and-providers
in order to make it more suitable for managing a jenkins deployment.

Configuration of Jenkins jobs is handled via the
\texttt{job-dsl}FOOTNOTE:URL-https://plugins.jenkins.io/job-dsl plugin.  This
enables a \texttt{groovy}FOOTNOTE:url-http://www.groovy-lang.org/ based DSL and a special job type that
will ``seed'' jenkins jobs from a git repository that contains \texttt{job-dsl}
script(s).  The seed job itself is maintained as \texttt{xml} that is manually
installed by puppet.  The result is that all jobs configuration is managed via
a SCM and no manual configuration via the jenkins UI is required to
add/delete/change or stand up a testing environment.

% should any of the jenkins related repos be mentioned?
% the production jenkins deployment is not publically accessible

% add discussion of jenkins pipeline

\subsubsection{Evolving Usage}

Over time, usage of jenkins has evolved from being used solely for pre-merge
``CI'' testing to automation of number of common tasks.  Among various sundry
tasks, it is being used to build docker images, update local software mirrors,
schedule regular backup.  Perhaps most notably, it is being used to a drive a
`Continuous Deployment'' workflow for completely automated nightly and weekly
release/publication of science pipelines codes(\ref{sec:scipipe-deploy}).

\subsection{Travis-CI}
\label{sec:travis-ci}

Travis for simpler items (yaml file validation, flake8 tests)

\section{Continuous Deployment}

\subsection{Deploying Science Pipelines Releases}

Weekly and daily releases. Publish to eups server and docker container.

\subsection{Case Study: Data Access Services}

The Data Access Services are a subset of the online services provided by the LSST Science Platform,\cite{LSE-319} providing
access to catalog and image data within the LSST data releases as well as some management of user-generated
data products.  In general the services are multi-user, scalable, and designed to be deployed in a data
center context rather than on individual developer machines.

The various components of the Data Access Services are built within LSST CI (Jenkins) as Docker containers.
Per general LSST practice, unit tests included in each module are executed as part of the build, and the build
will be fail if these do not pass.  In addition, some parts of the service suite have coverage by small-
scale automated integration tests; these tests are run by Travis CI on each GitHub branch commit or pull
request.  The automated integration tests launch a constellation of containers configured with test datasets,
then make a series of service requests and evaluate received responses against known/expected results.

The Data Access Services are at present deployed at scale for testing and development on three compute
clusters: one thirty node cluster at NCSA, and two twenty-five node clusters at CC-IN2P3.  These deployments
host a combination of synthetic test data and scientifically valid test data sourced from other astronomical
surveys such as SDSS, WISE, and HSC.  Scale testing is re-provisioned annually on clusters of increasing size
with datasets of increasing size, on a ramp toward the scale necessary to support start of operations.

Cluster deployments of the Data Access Services containers are done at present with a collection of ad-hoc
administration scripts, but the project is working currently to replace these with Kubernetes tooling.


\subsection{Case Study: Deploying releases to JupyterLab}

Link to main JupyterLab paper by Frossie et al.

\subsection{Deployment at IN2P3}


\section{Conclusion}

The LSST Data Management Team consists of more than one hundred people spread across multiple locations.
In this paper we have described our current developer processes and explained how they have evolved over time and how we foresee them evolving in the future.
The LSST Data Management software development has been ongoing for at least 14 years and we hope that our processes will continue to evolve, based on experience and new technologies, as we transition from construction to operations through to the end of the survey in 2032.


\acknowledgments % equivalent to \section*{ACKNOWLEDGMENTS}

We thank all the people who have contributed to the tooling, processes, and discussions  over the years.
This material is based upon work supported in part by the National Science Foundation through Cooperative Agreement 1258333 managed by the Association of Universities for Research in Astronomy (AURA), and the Department of Energy under Contract No.\ DE-AC02-76SF00515 with the SLAC National Accelerator Laboratory.
Additional LSST funding comes from private donations, grants to universities, and in-kind support from LSSTC Institutional Members.
This work includes a contribution funded by France's CNRS / IN2P3 (Institut National de Physique Nucl\'{e}aire et de Physique des Particules).

% References
\bibliography{spie-10707-10} % bibliography data
\bibliographystyle{spiebib} % makes bibtex use spiebib.bst

\end{document}
