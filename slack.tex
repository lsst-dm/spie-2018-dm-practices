\subsection{Slack}

The data management group uses instant messaging tools to answer questions, 
share code snippets and errors, and coordinate activities in real-time.  
Before using instant messaging, we would communicate primarily via e-mail or 
over the phone. While these do still have their place, being able to have 
conversations in a team-wide tool has been very beneficial. Discussions happen 
with more team members participating, instead of just who happened to be on a 
phone call or an e-mail thread. Questions and concerns have been able to be 
answered much more quickly.

We have used two different instant messaging tools in the last four years. The
first tool we used was HipChat. Besides the essential features one would
expect from an instant messaging client (such as discussion channels and one
to one direct messaging), we were also able to receive Jenkins and JIRA issue
notifications, which we found very helpful.

We used HipChat for two years and eventually started looking at other options 
because we found the user experience was, at times, inconsistent and 
frustrating. One of the most significant issues was how it managed unread
messages. If the user left the application, HipChat did not maintain where the
user was in the message chain. When the tool restarted, it required the user
to scroll back through messages to pick up where they left off. Notifications
were erratic, and sometimes would arrive late or not at all. Some users
particularly liked features such as video chat, while other users found this
feature unstable and resorted to using other video chat services such as Google 
Hangouts. 

After a brief testing period, we decided to transition to Slack, which we
continue to use today. The move from HipChat to Slack was straightforward,
although we did lose the discussion history from HipChat, and Slack initially
didn't have video chat (which it does now). We also took the opportunity to
normalize the users' chat names to match the names used on GitHub, which
simplified how we notify users of software build failures.

Slack offers of HipChat's functionality and improves several features. Slack 
has several options on how unread messages are handled and displayed, allowing
users to choose how the application behaves when the application restarts. 
Users can decide how to handle unread messages either showing them where the
application was last used, or by marking them already read.

Other features include direct messaging can be set up for small groups,
instead of creating a public chat room for an ephemeral discussion topic. 
Previews of URLs, global and per channel notification preferences and an 
API to integrate third-party or custom chat-bot apps. One of those apps, 
qrbot, is discussed in the next section.  

Other subsystems within LSST which either had already been using HipChat or
were beginning to explore instant messaging services have also transitioned to 
Slack. We expect this to be of great benefit to us as we integrate the work 
from the LSST sub-systems.

\subsubsection{Chat Ops}

DM-SQuaRE is using a Slack chatbot called ``sqrbot'' to make some tasks
easier. Currently it performs a range of functions, from returning the
status of various infrastructure machines to creating technotes to
monitoring whether metrics in our processing stack have changed between
CI builds. What these tasks have in common is that they were frequently
requested actions that required someone being asked for help and having
to interrupt their own work to perform a task. With sqrbot, the
requestor can simply ask sqrbot in a Slack channel for the information
(or creation of a technote skeleton, etc.) and get immediate gratification.

The basic architecture is simple: sqrbot is simply a collection of hubot
scripts running as a Slack bot, which in turn drive microservices,
written in Python and implemented using the Flask framework.  These
microservices have an API that responds with JSON, so the job of sqrbot
is simply to accept commands, create appropriate HTTP transactions, and
then reformat the output into a conversational format.  The whole
assembly runs in a Kubernetes cluster.

