\subsection{Slack}
\label{sec:slack}

The data management group uses instant messaging tools  to answer questions,
share code snippets and errors, and coordinate activities in real-time.
Before using instant messaging, we would communicate primarily via e-mail or
over the phone. While these tools do still have their place, being able to have
conversations in a team-wide application has been very helpful.
Site channels help coordinate discussions
useful to local teams.  We have non-work related channels where we can
discuss various topics, which has sparked conversations about previously
unrealized common interests. It has been beneficial to team building and
helped foster friendships.


We used HipChat for two years and eventually started looking at other options
because we found the user experience was, at times, inconsistent and
frustrating — especially treatment of message history.
After a brief testing period, we transitioned to Slack.
The move from HipChat to Slack was straightforward,
although we did lose online access to the discussion history from HipChat.  At first, we
thought this would be an issue, but in practice, we rarely search history older
than a couple of weeks.  We have had difficulties in balancing when to
archive these discussions for the community.  We've also had some problems
making sure that decisions made within a Slack conversation were
distributed to more team members, since everyone may not have been in the
channel at that time.  We continue to improve our efforts in this regard by
starting RFC's or Confluence discussions to distribute information more widely.

Slack initially didn't have video chat when we switched from HipChat, but
it does now.  We use this for ad-hoc calls for the channel, which has been
easier than coordinating a call via Google Hangouts or BlueJeans.

Direct messaging can be set up for small groups,
instead of creating a public chat room for an ephemeral discussion topic.
We create temporary channels for members to communicate during workshops and
conferences, which helps team members coordinate while on-site and to involve
people who aren't attending.  This paper itself was even organized
within a Slack channel, allowing us to discuss it and to get automatic
notifications of changes to it via the GitHub Slack bot.
Other features include previews of URLs, global and per channel notification preferences, and an
API to integrate third-party and custom chat-bot apps. One of those apps,
sqrbot, is discussed in the next section.

\subsubsection{Chat Ops}

DM-SQuaRE is using a Slack chatbot called ``sqrbot'' to make some tasks
easier. Currently it performs a range of functions, from returning the
status of various infrastructure machines to creating technotes to
monitoring whether metrics in our processing stack have changed between
CI builds. What these tasks have in common is that they were frequently
requested actions that required someone being asked for help and having
to interrupt their own work to perform a task. With sqrbot, the
requestor can simply ask sqrbot in a Slack channel for the information
(or creation of a technote skeleton, etc.) and get immediate gratification.

The basic architecture is simple: sqrbot is simply a collection of hubot
scripts running as a Slack bot, which in turn drive microservices,
written in Python and implemented using the Flask framework.  These
microservices have an API that responds with JSON, so the job of sqrbot
is simply to accept commands, create appropriate HTTP transactions, and
then reformat the output into a conversational format.  The whole
assembly runs in a Kubernetes cluster.

